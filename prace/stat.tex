\chapter{Požadavky na řešení}
Na celé řešení mám několik požadavků, které postupně detailně popíši tak, aby bylo vše jasné. Poněvadž není nic horšího 
než nepřesné, či nejasné zadání, protože se pak výsledek špatně hodnotí a celkově upřesňování zadání v průběhu řešení je 
cesta do pekel.

První požadavek se bude týkat měření. Měřit chci teplotu a vlhkost v teráriu, když bude zvolený hardware umět i něco 
jiného, klidně to použiji, ale hlavní požadavek je na tyto dvě veličiny. Ohledně frekvence měření myslím, že pro mé 
účely bude stačit měřit jednou za čtvrt hodiny, to je 96 měření za den. Neodchytím tím sice drobné výkyvy v rámci minut, 
ale cílem je zachytit dlouhodobý průběh hodnot v rámci dne, kdy mne tak drobné výkyvy nezajímají. Pro tento účel by se 
čtvrt hodiny mohlo zdát možná až jako příliš často, ale já bych to tak nechal z důvodu zjemnění denních grafů, přeci 
jenom graf se třeba 12 hodnotami nevypadá úplně nejlépe, a také z důvodu odchycení případných chyb měření nebo náhlých 
změn, například zalévání\ldots
\chapter{Analýza problému}
\chapter{Hardware}
\chapter{Software}
\section{Měřící stanice}
\section{Domácí gateway}
\section{Cloud}
\section{Zobrazení grafů}
\chapter{Výsledek}
