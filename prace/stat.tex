\chapter{Požadavky na řešení}
Na celé řešení mám několik požadavků, které postupně detailně popíši tak, aby bylo vše jasné. Poněvadž není nic horšího 
než nepřesné, či nejasné zadání, protože se pak výsledek špatně hodnotí a celkově upřesňování zadání v průběhu řešení je 
cesta do pekel.

První požadavek se bude týkat měření. Měřit chci teplotu a vlhkost v teráriu, když bude zvolený hardware umět i něco 
jiného, klidně to použiji, ale hlavní požadavek je na tyto dvě veličiny. Ohledně frekvence měření chci zachytit denní 
trendy, ale nepotřebuji data z každé minuty.

Další z požadavků je na ukládání dat. Když už je změřím, tak je chci mít vždy uložená, tedy i při výpadku internetu 
a podobně. Při výpadku proudu nic nezměřím, takže to není třeba řešit. Co se týče vzdáleného ukládání, nepovažuji za 
důležité, aby se všechna data propsala do cloudu, tedy i v případě nějakého výpadku se odeslala data co jsem změřil, ale 
neodeslal. Když přijdu o jedno měření během krátkého výpadku, je mi to jedno a při větším nebo nějaké chybě si toho 
všimnu a to stejně bude třeba opravit.

Nároky na uživatelské rozhraní v podstatě nemám. Pro správu senzorů je nepotřebné. Stejně moc obměňovat, či přidávat 
nebudu, a i kdyby Stejně si budu muset napsat obslužný program pro ten či onen. Udělat nějaký obalující systém pro více 
senzorů, či knihovnu není mým cílem. Pro zobrazení hodnot je důležité, ale ohledně nároků mi bude stačit velmi 
jednoduché, třeba jen s jediným grafem, bez možnosti změnit časový rámec\ldots Avšak případnému rozšíření se nebráním. 
Další části u nichž by bylo třeba komunikovat s uživatelem mě nenapadají. Možná správa uživatelů s přístupem, ale 
vzhledem k tomu, že to dělám pro sebe bych to vynechal.

K bezpečnosti bych rád zmínil. Bylo by dobré mít komunikaci po celé komunikační trase šifrovanou, ale dokud řešení 
neobsahuje ovládání čehosi, není to úplně nezbytné. Případný útočník by si tak maximálně zjistil vlhkost\ldots v teráriu
nebo by se mohl teoreticky vydávat za teploměr a kazit mi data. To by mohl být problém, takže pokud nevyžaduji 
šifrování, ověření toho kdo posílá data požaduji určitě. Zabezpečení však považuji za důležité u samotné webové stránky, 
která bude vystavena veřejně. Ne že by mi vadilo, že někdo sleduje jak se mají mé kytičky, ale mohlo by se z toho dát 
odvodit, že je nezalévám tj. jsem pryč a v bytě nikdo není. Pokud si někdo dá tu práci, že mi napíchne připojení, nebo 
přijede a odposlechne data z domácí sítě, tak má asi i možnosti jak si to, že nejsem doma zjistit jinak. Rozhodně to je 
složitější než otevření webové stránky.
\chapter{Analýza problému}
% TODO nějak to obal
Ohledně frekvence měření myslím, že pro mé účely bude stačit měřit jednou za čtvrt hodiny, to je 96 měření za den. 
Neodchytím tím sice drobné výkyvy v rámci minut, ale cílem je zachytit dlouhodobý průběh hodnot v rámci dne, kdy mne tak 
drobné výkyvy nezajímají. Pro tento účel by se čtvrt hodiny mohlo zdát možná až jako příliš často, ale já bych to tak 
nechal z důvodu zjemnění denních grafů, přeci jenom graf se třeba 12 hodnotami nevypadá úplně nejlépe, a také z důvodu 
odchycení případných chyb měření nebo náhlých změn, například při zalévání atd.
Takže pokud by to bylo možné by bylo nejlepší ukládat hodnoty přímo v místě měření.


\chapter{Hardware}
\chapter{Software}
\section{Měřící stanice}
\section{Domácí gateway}
\section{Cloud}
\section{Zobrazení grafů}
\chapter{Výsledek}
