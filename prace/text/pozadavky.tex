% úvod
%Na celé řešení mám několik požadavků, které postupně detailně popíši tak, aby bylo vše jasné. Poněvadž není nic horšího 
%než nepřesné, či nejasné zadání, protože se pak výsledek špatně hodnotí a celkově upřesňování zadání v průběhu řešení 
%je cesta do pekel. Tyto požadavky jsou nutné, pokud vytvořené řešení bude mít funkce navíc, nevadí to, ale nesmí 
%ovlivňovat tyto zadané parametry.

% měření
První požadavek se bude týkat měření. Měřit chci teplotu a vlhkost v teráriu, když bude zvolený hardware umět i něco 
jiného, klidně to použiji, ale hlavní požadavek je na tyto dvě veličiny. Ohledně frekvence měření chci zachytit denní 
trendy, ale nepotřebuji data z každé minuty.

% ukládání
Další z požadavků je na ukládání dat. Když už je změřím, tak je chci mít vždy uložená, tedy i při výpadku internetu 
a podobně. Při výpadku proudu nic nezměřím, takže to není třeba řešit. Co se týče vzdáleného ukládání, nepovažuji za 
důležité, aby se všechna data propsala do cloudu, tedy i v případě nějakého výpadku se odeslala data co jsem změřil, ale 
neodeslal. Když přijdu o jedno měření během krátkého výpadku, je mi to jedno.% a při větším nebo nějaké chybě si toho 
%všimnu a stejně to bude třeba opravit.

% analýza dat
Data mám uložena, co s nimi budu dělat? No tak v podstatě bych nemusel dělat nic, pouze si je zobrazit, což je to co 
požaduji. Avšak hezká by byla možnost zobrazit si nějaké pokročilejší statistiky. Takže volitelně přidávám požadavek na 
zobrazení různých časových rámců a statistiky k onomu časovému období, například průměrná, nejvyšší, nejnižší či medián 
teploty a vlhkosti. Případně různé tendence, derivace, co bude v možnostech vybraného nástroje.

% UI
Nároky na uživatelské rozhraní v podstatě nemám. Pro správu senzorů je nepotřebné. Stejně moc obměňovat, či přidávat 
nebudu, a i kdyby Stejně si budu muset napsat obslužný program pro ten či onen. Udělat nějaký obalující systém pro více 
senzorů, či knihovnu není mým cílem. Pro zobrazení hodnot je důležité, ale ohledně nároků mi bude stačit velmi 
jednoduché, pokud možno jediná stránka. Avšak případnému rozšíření se nebráním. Další části u nichž by bylo třeba 
komunikovat s uživatelem mě nenapadají. Možná správa uživatelů s přístupem, ale vzhledem k tomu, že to dělám pro sebe 
bych to vynechal.

% bezpečnost
K bezpečnosti bych rád zmínil toto. Bylo by dobré mít komunikaci po celé komunikační trase šifrovanou, ale dokud řešení 
neobsahuje ovládání čehosi, není to úplně nezbytné. Případný útočník by si tak maximálně zjistil vlhkost\ldots v teráriu
nebo by se mohl teoreticky vydávat za teploměr a kazit mi data. To by mohl být problém, takže pokud nevyžaduji 
šifrování, ověření toho kdo posílá data požaduji určitě. Zabezpečení však považuji za důležité u samotné webové stránky, 
která bude vystavena veřejně. Ne že by mi vadilo, že někdo sleduje jak se mají mé kytičky, ale mohlo by se z toho dát 
odvodit, že je nezalévám tj. jsem pryč a v bytě nikdo není.% Pokud si někdo dá tu práci, že mi napíchne připojení, nebo 
%přijede a odposlechne data z domácí sítě, tak má asi i možnosti jak si to, že nejsem doma zjistit jinak. Rozhodně to 
%ale bude složitější než otevření webové stránky.

% cena
%Pak tu mám poslední z požadavků a tím je cena. Nejraději bych ho neřešil, ale žijeme ve světě kde je tento požadavek 
%velmi důležitý. Takže asi jediné cenové omezení je, abych si to mohl dovolit. Tady bych zopakoval, že cenu budu 
%částečně upřednostňovat před ostatními požadavky. Zejména v případech, kdy mám nějaký hardware doma, i když ne úplně 
%vhodný, ale dostačující. Tak ho upřednostním před vhodnějším, který nemám a musel bych ho draze shánět. Otázku ceny 
%budu podrobněji řešit u výběru hardwaru.
