% slovník
\newglossaryentry{go}
{
  name={Go},
  description={Kompilovaný, staticky typovaný jazyk od Googlu, a na jehož vývoji se podílel například Ken Thompson, 
spolutvůrce programovacího jazyka C, z jehož syntaxe vychází i syntaxe Go. Cíle jazyka jsou zejména jednoduchá syntax, 
strmá křivka učení, či snadná tvorba vícevláknových aplikací. Kompromisem v návrhu jazyka bylo zahrnutí \gls{GC}, 
programy jsou sice pomalejší, ale kód jednodušší. Jazyk je oblíben i mimo Google, je v něm napsán například Docker, či 
ho používá Dropbox.}
}
\newglossaryentry{GC}
{
  name={garbage collector},
  description={Způsob automatické správy paměti. Funguje tak, že speciální algoritmus (garbage collector) vyhledává 
a uvolňuje úseky paměti, které již program nebo proces nepoužívá. Programátor to tedy již nemusí řešit a tím odpadá celá 
řada chyb způsobených zapomenutím na uvolnění paměti\ldots Nevýhodou je, že si garbage collector ukousne část výkonu 
procesoru pro sebe, takže pak program běží pomaleji.}
}
\newglossaryentry{cross-kompilace}
{
  name={cross-kompilace},
  description={Překlad programu pro jinou platformu, než na které je překládán.}
}
\newglossaryentry{linkovani}
{
  name={linkování},
  description={Proces spojení objektového souboru vygenerovaného překladačem s knihovnami, či jinými soubory. Existují 
dva typy dynamické, kdy se knihovny připojují až za běhu a jsou společné pro všechny programy běžící na daném počítači 
a statické, kdy jsou všechny knihovny přibaleny k výslednému spustitelnému souboru.}
}
\newglossaryentry{knihovna}
{
  name={knihovna},
  description={Soubor funkcí, tříd, či konstant, které jsou využívány více programy.}
}
\newglossaryentry{logrus}
{
  name={logrus},
  description={Logrus je taková rozšířená verze \glslink{knihovna}{knihovny} log, ze standartu jazyka. Umožňuje mi 
detailně logovat běh programu a strukturovat logy, například tím, že ke každé zprávě kterou vypíši, mi doplní důležité 
informace, jako z jaké funkce byl zavolán, na jakém řádku, nebo čas kdy byl zavolán. Toto se dá libovolně měnit. A navíc 
díky němu mohu nechat ladící hlášky v programu celou dobu, jenom na začátku v inicializaci mu řeknu, že mě zajímají 
pouze chyby a on mi hlášky s nižší prioritou, tj. debugovací informace a podobně nebude vypisovat. Takže po nasazení, si 
jen zvolím soubor, kam má logy ukládat, abych o ně nepřišel a prioritu co má vypisovat, a pak můžu jen sledovat chyby.}
}
\newglossaryentry{periphio}
{
  name={periph.io},
  description={Knihovna pro jazyk \gls{go}, která se snaží zapouzdřit komunikaci po \acrshort{gpio} a poskytuje rozhraní 
pro oblíbené senzory, aby jejich použití bylo co nejjednodušší.}
}
\newglossaryentry{timestamp}
{
  name={timestamp},
  description={Počet sekund/něco s větší přesností od 1.1. 1970}
}
\newglossaryentry{brana}
{
  name={brána},
  description={Centrální bod, kam senzory s celé domácnosti odesílají data. Zde by se rozhoduje co s nimi a případně se 
  posílají dál. Případné akční prvky (například žárovka) by braly informace odtud.}
}
\newglossaryentry{ssl}
{
  name={SSL},
  description={Protokol vytvářející šifrovanou cestu mezi transportní a aplikační vrstvou. Dnes se místo něj používá 
protokol TLS, avšak stále se používá označení SSL.}
}
\newglossaryentry{certifikat}
{
  name={certifikát},
  description={Digitálně podepsaný veřejný šifrovací klíč, kterým někdo nebo něco dokazuje, že je tím za koho se 
vydává.}
}
\newglossaryentry{message-broker}
{
  name={message broker},
  description={Server zajišťující komunikaci, obvykle se používá právě v IoT nebo třeba v distribuovaných systémech. 
Funguje na principu tzv. témat asi takto, já mu pošlu zprávu s určitým tématem (názvem), obsahující zrovna třeba 
naměřené teploty a on ji přepošle všem, kdo jsou přihlášeni k odběru zpráv daného tématu. Může se stát, že se jednotlivé 
části v různých programech jmenují jinak, ale princip je stejný. Toto je takzvaná Publish-Subscribe strategie, její 
vlastnost však je, že server zprávu pošle a pak ji zahodí, takže pokud někomu nepřijde, třeba z důvodu výpadku 
proudu\ldots tak už ji nikdy nedostane. Někomu to může vadit, takže pak vznikají nadstavby, kdy server zprávu uloží 
a zkouší ji poslat, dokud od klienta neobdrží potvrzení o přijetí, to je mnohem robustnější řešení.}
}
\newglossaryentry{JSON}
{
  name={JSON},
  description={Datový formát založený na syntaxi javascriptu, oblíbený zejména ve webových aplikacích.}
}
\newglossaryentry{gorutina}
{
  name={gorutina},
  description={Odlehčené vlákno}
}
\newglossaryentry{I2C}
{
  name={I$^2$C},
  description={Sériová sběrnice od firmy Philips, používaná pro připojení nízko rychlostních periferií. Používá čtyři 
dráty: napětí, zem, hodiny a datový kabel.}
}
\newglossaryentry{onewire}
{
  name={OneWire},
  description={Sériová sběrnice používaná výhradně firmou Dallas, pro připojení jejích produktů. Stačí jí pouze tři 
dráty: napětí, zem a datový s tím, že lze odstranit napájecí kabel a senzor je poté napájen paraziticky s datového 
kabelu.}
}

% akronymy
\newacronym{gpio}{GPIO}{general purpose input/output}
\newacronym{csv}{CSV}{comma separated values}
