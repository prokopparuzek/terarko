\newglossaryentry{go}
{
  name={Go},
  description={Kompilovaný, staticky typovaný jazyk od Googlu, a na jehož vývoji se podílel například Ken Thompson, 
spolutvůrce programovacího jazyka C, z jehož syntaxe vychází i syntaxe Go. Cíle jazyka jsou zejména jednoduchá syntax, 
strmá křivka učení, či snadná tvorba vícevláknových aplikací. Kompromisem v návrhu jazyka bylo zahrnutí \gls{GC}, 
programy jsou sice pomalejší, ale kód jednodušší. Jazyk je oblíben i mimo Google, je v něm napsán například Docker, či 
ho používá Dropbox.}
}
\newglossaryentry{GC}
{
  name={garbage collector},
  description={Způsob automatické správy paměti. Funguje tak, že speciální algoritmus (garbage collector) vyhledává 
a uvolňuje úseky paměti, které již program nebo proces nepoužívá. Programátor to tedy již nemusí řešit a tím odpadá celá 
řada chyb způsobených zapomenutím na uvolnění paměti\ldots Nevýhodou je, že si garbage collector ukousne část výkonu 
procesoru pro sebe, takže pak program běží pomaleji.}
}
\newglossaryentry{cross-kompilace}
{
  name={cross-kompilace},
  description={Překlad programu pro jinou platformu, než na které je překládán.}
}
\newglossaryentry{linkovani}
{
  name={linkování},
  description={Proces spojení objektového souboru vygenerovaného překladačem s knihovnami, či jinými soubory. Existují 
dva typy dynamické, kdy se knihovny připojují až za běhu a jsou společné pro všechny programy běžící na daném počítači 
a statické, kdy jsou všechny knihovny přibaleny k výslednému spustitelnému souboru.}
}
