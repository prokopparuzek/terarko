% výběr jazyka
Jako jazyk pro programování měřící stanice, jsem zvolil \gls{go}. Jedna z mnoha výhod je snadná \gls{cross-kompilace} 
která mi umožňuje kompilovat programy na svém počítači a do stanice nahrávat už jen binární kód. Čemuž pomáhá i to, že 
překladač implicitně \glslink{linkovani}{linkuje} staticky.

% koncepce programu
Obecná koncepce programu je asi takováto. Budu sledovat běh funkce main, kterou jsem se snažil co nejvíce vyčistit.
\lstinputlisting[language=go,breaklines=true]{src/measure.go}
Na začátku importuji šest \glslink{knihovna}{knihoven}, konkrétně používám části standardní \glslink{knihovna}{knihovny} 
pro práci s časem a operačním systémem, \glslink{knihovna}{knihovnu} pro práci s periferiemi a dále to jsou 
\glslink{knihovna}{knihovny} pro spojení s bránou, umožnění automatického spouštění programů v daný čas a logování. Ve 
funkci main, jako první nastavím \glslink{knihovna}{knihovnu} logrus, který mi zajišťuje logování, nastavuji co chci 
logovat, kam to má zapsat a jak to má zformátovat. Poté otevřu spojení s bránou, inicializuji 
\glslink{knihovna}{knihovnu} na použití periferií a nastavím cron, aby mi spustil měření každých 15 minut. Nakonec je 
takový trik, jehož jediným účelem je, aby hlavní gorutina neskončila, jedná se o čtení z kanálu do kterého, však nikdy 
nic nezapíši. A jelikož jde o blokující operaci, program nikdy neskončí.

% měření
\paragraph*{Měření}
Samotné získávání dat je velmi jednoduché. Každých 15 minut se zavolá funkce getMeasure(),která postupně projde přes 
všechny připojené senzory, zavolá funkce, které je obsluhují, v případě chyby je zkouší zavolat vícekrát a vrátí pole 
s naměřenými hodnotami. Pro měření jsem se snažil co nejvíce využít možností, které poskytuje  \gls{knihovna} 
\gls{periphio}, takže například data vracím ve formě struktury z této knihovny a používám ji pro získávání hodnot ze 
dvou senzorů, pro třetí ji nepoužívám jen z toho důvodu, že ho zatím nepodporuje. Měření z BME280 je velmi jednoduché. 
V podstatě otevřu sběrnici, přečtu data a vrátím je, vše za použití výše zmíněné knihovny. Z DS18B20 to je velmi podobné, 
jen nemohu použít \gls{periphio}, takže používám knihovnu, jež využívá modul kernelu, který zpřístupňuje tento senzor 
přes virtuální souborový systém. Senzor DHT11 je na použití asi nejsložitější. Používám sice externí knihovnu, avšak ta 
pro přístup ke GPIO využívá \gls{periphio}. Jinak je měření velmi obdobné jako u ostatních čidel, s tím rozdílem, že je 
velmi chybové, takže se musí vícekrát opakovat.

\paragraph*{DHT11}
S tímto senzorem jsem měl asi největší problémy. Nejenom že jsem musel laborovat s nastavením verze api knihovny 
periph.io, ale i samotná knihovna pro obsluhu senzoru obsahovala chyby. Takže jsem ji důkladně prozkoumal a porovnal 
s datasheetem k senzoru. Našel jsem dvě zásadní chyby. Jedna byla, že knihovna vyslal příliš krátký startovací pulz, 
takže ani čidlo neprobudila, to se dalo vyřešit jednoduše, prostě jsem do programu přidal pauzu. A druhá, že špatně 
zpracovávala data co z čidla četla. Před úpravou mi vracela nereálné hodnoty teploty a vlhkosti, konkrétně vracela 
chybu, že načtená data jsou mimo rozsah, protože hodnoty zpracovávala jako jedno velké šestnáctibitové číslo, ale 
v datasheetu jsem se dočetl že tyto dva bajty představují číslo v desetinné čárce, ale velmi neobvykle. První bajt je 
celá část a druhý desetinná a zároveň s tím se v datasheetu píše že rozlišovací schopnost senzoru je 
1 \textdegree C a 1 \%, takže jsem druhý bajt zahodil a dál pracoval pouze s tím prvním. A to fungovalo na jedničku. 
Tuto zvláštnost s rozlišením bych asi vysvětlil tím, že komunikační protokol bude schodný i s dražšími senzory s lepším 
rozlišením a to možná souvisí i s chybami v knihovně, poněvadž je určena i pro ně. Takže abych jejich podporu nerozbil 
jsem mé úpravy podmínil použitím správného typu čidla. Moje upravená verze kterou používám, je k nalezení zde: 
\href{https://github.com/prokopparuzek/go-dht.git}{github.com/prokopparuzek/go-dht.git}

% ukládání dat
\paragraph*{Ukládání}
Pro ukládání dat na měřící stanici jsem nevymýšlel nic složitého. Vezmu jen data z každého senzoru, přidám 
\gls{timestamp}, tím se vyhnu problémům s reprezentací času a převody časových pásem a to vše přidám za konec souboru 
konkrétního senzoru. Data ukládám ve formátu \acrshort{csv}. Jednotlivé hodnoty/sloupce nijak neoznačuji, nechávám to na 
utilitách, jejichž cílem bude data obnovit, ty jediné s nimi takto budou pracovat a to jen občas, takže to myslím 
nevadí, a navíc to zjednodušuje kód.

\paragraph*{Message broker}
Jako základní \gls{message-broker} by se dal použít server NATS. Já použiji trochu bezpečnější variantu, konkrétně 
použiji nadstavbu nazývanou NATS-streaming. Stejně jako NATS se jedná o lehkou aplikaci napsanou v go, takže zabírá 
minimum zdrojů. Já jsem ji vybral z důvodu jednoduchosti, dostupnosti široké škály klientských knihoven a též již 
zmiňované lehkosti. A taky proto že mám rád go.

% odesílání
\paragraph*{Odesílání}
Odesílání dat začíná vlastně už na úplném začátku, kdy se spojím se serverem a pak až do konce držím spojení otevřené. 
Samotné odeslání dat do \glslink{message-broker}{message brokera}, se vlastně moc neliší od uložení. Taky vezmu data, 
přidám \gls{timestamp} a pošlu je. Je tu však pár rozdílů. Data posílám ve formátu \gls{JSON}, který je jednoduchý 
a čitelný. Z důvodu možných latencí, nedostupnosti sítě\ldots odesílám každou zprávu v samostatné 
\glslink{gorutina}{gorutině}, abych neblokoval další měření, jelikož když se zprávu nepovede odeslat, tak ji zkouším 
poslat po minutě další zhruba dvě hodiny. No a to je vše po odeslání zprávy už nemusím nic řešit, poněvadž server si ji 
uloží a pošle dál, takže už se neztratí.
