% úvod
Zde bych rád rozebral jak budu jednotlivé požadavky řešit. Nastínil jak bude celkové řešení vypadat, strukturu\ldots Též 
bych zde rád představil kde vlastně budu měřit a zpracovávat data.

% měření
Ohledně měření bych zatím zmínil jen frekvence o které myslím myslím, že pro mé účely bude stačit měřit jednou za čtvrt 
hodiny, to je 96 měření za den. Neodchytím tím sice drobné výkyvy v rámci minut, ale cílem je zachytit dlouhodobý průběh 
hodnot v rámci dne, kdy mne tak drobné výkyvy nezajímají. Pro tento účel by se čtvrt hodiny mohlo zdát možná až jako 
příliš často, ale já bych to tak nechal z důvodu zjemnění denních grafů, přeci jenom graf se třeba 12 hodnotami nevypadá 
úplně nejlépe, a také z důvodu odchycení případných chyb měření nebo náhlých změn, například při zalévání atd.

% ukládání
Co se týče uložení dat, abych zajistil stoprocentní jistotu, že se data uloží, budu je ukládat přímo v místě měření, tím 
myslím, že počítač, který bude mít na starosti měřit, bude zároveň naměřená data hned ukládat na SD kartu, nebo tak 
někam. Bude to takový log měření, který mi umožní obnovit data nezapsaná do cloudu z důvodu nějaké chyby, případně mi 
umožní zjišťovat kde vlastně chyba nastala, a může pomoci i s řešením. Dále myslím, že data bude stačit ukládat klidně 
pouze na jedno místo v cloudu, poněvadž ten sám o sobě pokud je kvalitní je výborně zálohovaný, pokud o data v něm 
nějakým nedopatřením přijdu, nějak mě to neovlivní, bude to škoda, avšak závažné důsledky to mít nebude a navíc tím, že 
nebudu řešit zálohy a rozkopírovávání dat, či jejich konzistenci na různých místech se celé řešení výrazně zjednoduší.
