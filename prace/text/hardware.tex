% úvod
%Konečně se dostáváme k něčemu reálnému. Zde popíši hardwarovou část řešení tj. Co vlastně bude měřit, čím budu 
%měřit\ldots

% měřící stanice
\section{Měřící stanice}
\begin{figure}[H]
    \centering
    \includegraphics[width=0.8\textwidth]{RPi2.jpg}
    \caption{Raspberry pi 2}
\end{figure}
Možných základů pro měřící stanici, je dnes na trhu mnoho, od různých osmibitů až po počítače na architektuře ARM,  
které dosahují výkonu srovnatelného s mobilními telefony \parencite{wiki:raspberrypi}. Já jsem pro mé řešení zvolil 
Raspberry Pi ve verzi 2 a to z několika důvodů. Jednak ho mám k dispozici a dále mi nabízí běžící Linux a tudíž za mne 
řeší spoustu problémů, od síťové komunikace po synchronizaci času. Navíc mám k dispozici spoustu digitálních pinů pro 
připojení různých senzorů. Též mám vyřešené i místní úložiště, data se mohou ukládat na SD kartu ze které běží celý 
systém. Výrazně to zjednodušuje vývoj, poněvadž mohu nahrávat nové verze programů vzdáleně, a i vzdáleně sledovat jejich 
běh, což je pro mě výhodné, neb terárium nemám v pokoji, kde programuji. Samozřejmě že toto řešení má i své nevýhody. 
Například v případě, že bych chtěl měřit analogové hodnoty, bych musel dokupovat převodník z analogového signálu na 
digitální. V případě nutnosti rozšíření na více míst, by to nebylo ekonomicky výhodné, přeci jen Raspberry Pi stojí 
kolem 1000 Kč. Nebo pokud bych chtěl zařízení napájet z akumulátoru, tak s odběrem kolem půl ampéru by moc dlouho 
nevydržel \parencite{wiki:raspberrypi}. Avšak v případě zmíněných problémů mi zvolené celkové řešení umožňuje poměrně 
komfortně změnit základ stanice na něco vhodnějšího za předpokladu, že se zvolená deska zvládne připojit na lokální síť. 
Například mohu použít oblíbenou desku ESP8266 či ESP32, které se cenově pohybují v řádu stokorun, pinů mají dostatek, 
disponují Wifi čipem a umožňují použití nízko odběrových módů při běhu na baterii \parencite{root.cz:ESP}.

% senzory
\section{Senzory}
% BME280
\subsection{BME280}
Pro měření hodnot v teráriu jsem zvolil senzor BME280 teploty vlhkosti a tlaku vzduchu od firmy Bosh s cenovkou kolem 
200Kč. Navíc díky sběrnici \gls{I2C} mi z terária vedou pouze čtyři dráty \parencite{pajenicko:BME280}.

\begin{figure}[H]
    \centering
    \includegraphics[angle=270,width=0.5\textwidth]{BME280.jpg}
    \caption{Modul s BME280, stříbrný čtvereček uprostřed}
\end{figure}

\begin{table}[H]
    \centering
    \begin{tabular}{|l|c|}
        \hline
        \multicolumn{2}{|c|}{Teplota} \\ \hline
        \hline
        Rozsah & -40 až +85\textdegree C \\ \hline
        Rozlišení & 0,01\textdegree C \\ \hline
        Přesnost & $\pm$ 1\textdegree C \\ \hline
        \hline
        \multicolumn{2}{|c|}{Vlhkost} \\ \hline
        \hline
        Rozsah & 0 až 100\% \\ \hline
        Rozlišení & 0,008\% \\ \hline
        Přesnost & $\pm$3\% \\ \hline
        \hline
        \multicolumn{2}{|c|}{Tlak} \\ \hline
        \hline
        Rozsah & 300 až 1100 hPa \\ \hline
        Rozlišení & 0,18 Pa \\ \hline
        Přesnost & 1 $\pm$ Pa \\ \hline
    \end{tabular}
    \caption{Parametry BME280}
\end{table}

% DS18B20
\subsection{DS18B20}
Za účelem případné další analýzy jsem umístil pár senzorů i mimo terárium, abych mohl sledovat závislost teploty vně 
a uvnitř.Jako hlavní senzor teploty jsem použil DS18B20 vyvinutý firmou Dallas s cenou čínské kopie asi 35 Kč. Jde o můj 
oblíbený senzor, poněvadž je přesný, snadno použitelný a díky sběrnici \gls{onewire} mu stačí pouze tři, případně dva 
dráty \parencite{pajenicko:DS18B20}.

\begin{figure}[H]
    \centering
    \includegraphics[angle=270,width=0.5\textwidth]{DS18B20.jpg}
    \caption{DS18B20, pouzdro TO-92}
\end{figure}

\begin{table}[H]
    \centering
    \begin{tabular}{|l|c|}
        \hline
        \multicolumn{2}{|c|}{Teplota} \\ \hline
        \hline
        Rozsah & -55 až +125\textdegree C \\ \hline
        Rozlišení & 0,0625\textdegree C \\ \hline
        Přesnost & $\pm$ 0,5\textdegree C \\ \hline
    \end{tabular}
    \caption{Parametry DS18B20}
\end{table}

% DHT11
\subsection{DHT11}
Pro měření vlhkosti vně terária jsem použil oblíbený senzor teploty a vlhkosti DHT11 s cenovkou kolem 40 Kč. Senzor 
teploty jsem zdvojil z důvodu velké nepřesnosti tohoto modelu. S tímto senzorem jsem měl největší problémy, zejména díky 
jeho nestandardní sběrnici, která připomíná \gls{onewire}, ale používá jiný komunikační protokol 
\parencite{pajenicko:DHT11}.

\begin{figure}[hbt]
    \centering
    \includegraphics[angle=270,width=0.5\textwidth]{DHT11.jpg}
    \caption{DHT11, modrý obdélníček}
\end{figure}

\begin{table}[H]
    \centering
    \begin{tabular}{|l|c|}
        \hline
        \multicolumn{2}{|c|}{Teplota} \\ \hline
        \hline
        Rozsah & 0 až +50\textdegree C \\ \hline
        Rozlišení & 1\textdegree C \\ \hline
        Přesnost & $\pm$ 2\textdegree C \\ \hline
        \hline
        \multicolumn{2}{|c|}{Vlhkost} \\ \hline
        \hline
        Rozsah & 20 až 90\% \\ \hline
        Rozlišení & 1\% \\ \hline
        Přesnost & $\pm$5\% \\ \hline
    \end{tabular}
    \caption{Parametry DHT11}
\end{table}

% gateway
\section{Brána}
Jako brána pro komunikaci s internetem se dá taky použít téměř cokoli, ale přeci jen jsou na ní kladeny větší nároky než 
na měřící stanici. Mimo nutnosti možnosti připojení do sítě je též třeba výpočetní výkon a paměť dostatečná na 
komunikaci s cloudem, řešení šifrování, běh nějakého \glslink{message-broker}{message brokera}\ldots, tedy nejlépe 
nějakou desku s operačním systémem, který mi tohle všechno umožní. Já jsem zvolil opět Raspberry Pi, tentokrát ve verzi 
3 opět z velmi jednoduchého důvodu, už mi doma běží, jako takový domácí server.
