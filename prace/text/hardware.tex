% úvod
Konečně se dostáváme k něčemu reálnému. Zde popíši hardwarovou část řešení tj. Co vlastně bude měřit, čím budu 
měřit\ldots

% měřící stanice
Možných základů pro měřící stanici, jednodeskových počítačů, je dnes na trhu mnoho od různých osmibitů, až po počítače 
na architektuře ARM,  které dosahují výkonu srovnatelného s mobilními telefony. Já jsem pro mé řešení zvolil Raspberry 
Pi ve verzi 2 a to z několika důvodů. Jednak ho mám k dispozici a dále mi nabízí běžící Linux a tudíž za mne řeší 
spoustu problémů, od síťové komunikace, po třeba synchronizaci času. Navíc mám k dispozici spoustu digitálních pinů pro 
připojení různých senzorů, též mám vyřešené i místní úložiště, data se mohou ukládat na SD kartu ze které běží celý 
systém a výrazně to zjednodušuje vývoj, poněvadž mohu nahrávat nové verze programů vzdáleně, a i vzdáleně sledovat 
jejich běh, což je pro mě výhodné, neb terárium nemám v pokoji, kde programuji. Samozřejmě že toto řešení má i své 
nevýhody. Například v případě, že bych chtěl měřit analogové hodnoty, bych musel dokupovat převodník z analogového 
signálu na digitální, či v případě nutnosti rozšíření na více míst, by to nebylo ekonomicky výhodné, přeci jen Raspberry 
Pi stojí kolem 1000 Kč. Nebo pokud bych chtěl zařízení napájet z akumulátoru, tak též s odběrem kolem půl ampéru to 
nebude to pravé ořechové. Avšak v případě zmíněných problémů mi zvolené celkové řešení umožňuje poměrně komfortně změnit 
základ stanice na něco vhodnějšího, za předpokladu, že se zvolená deska zvládne připojit na lokální síť. Například mohu 
použít oblíbenou desku ESP8266 či ESP32, které se cenově pohybují v řádu stokorun, pinů mají dostatek, disponují Wifi 
čipem a umožňují použití nízko odběrových módů při běhu na baterii.
