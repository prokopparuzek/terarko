Hlavní cíl mé práce tj. udržet kytičky naživu se myslím podařilo splnit, neb zatím žijí. Uvidím jak se jim bude dařit 
dál. Myslím že měření mi v lecčem pomohlo udržet je naživu, například jsem díky němu zjistil, že je málo zalévám, zvýšil 
jsem frekvenci a začali vypadat lépe.

Ohledně splnění cílů myslím, že vytvořené řešení splňuje až na jeden všechny hlavní požadavky, které jsme si zadal. 
Hodnoty měří, ukládá je na SD kartu a přes \glslink{brana}{bránu} je posílá do \gls{firebase}, odkud si je mohu zobrazit 
pouze já. Jediný požadavek, jež jsem nezvládl splnit je ten na podepisování posílaných dat, na ochranu před podvržením.

Myslím že případné zlepšování by se mohlo zaměřit na vylepšení zabezpečení cesty od senzorů na \glslink{brana}{bránu}, 
přidání statistik a možnosti zobrazení hodnot z většího intervalu, či zobrazení stavu senzorů přímo u terária. Případně 
by se dalo zrychlit přidávání nových senzorů, pokud možno tak že ho přidám jednou nastavím, kde je a co měří a to bude 
vše.
