Už potřetí zahajuji svůj pokus pěstovat masožravé rostliny, který zatím vždy skončil jejich úhynem. Z toho důvodu jsem 
se rozhodl začít sledovat prostředí v teráriu, kde je pěstuji, abych mohl v případě úhynu určit z jakého důvodu uhynuly. 
Přehřáli se, umrzli, uschly\ldots Většinou z důvodu mé nepřítomnosti, kdy jsem je nemohl kontrolovat. Avšak mnohem radši 
bych byl, kdyby se mi pomocí naměřených údajů podařilo udržet prostředí ve kterém prosperují a v případě náhlé změny 
mohl zasáhnout v krajním případě i vzdáleně.

Cílem mé práce je navržení systému pro měření v podstatě libovolných hodnot, jejich agregování na jednom místě, 
s možností zobrazení aktuálních dat, či jejich průběhu v minulosti, či navázáním různých alarmů na kritické hodnoty. 
Hodnoty by uživatel kontroloval s využitím webové aplikace, které zároveň zajistí snadnou použitelnost na mnoha 
platformách a přístupnost takřka na celém světě, tedy tam kde je internet.

Výsledkem práce bude samotná realizace řešení, od výběru hardwaru a dalších věcí jako je databáze\ldots po samotné 
sestavení měřícího zařízení, jeho naprogramování a naprogramování aplikace na zobrazení naměřených dat. Výsledný produkt 
by měl být snadno použitelný a rozšiřitelný o další funkce, možný budoucí vývoj je až aplikace na řízení tzv. chytrého 
domu. Z tohoto důvodu bude kladen důraz i na zabezpečení, pro zamezení neoprávněného přístupu. Z důvodů urychlení 
a zlevnění vývoje, nebudu vždy používat nejvhodnější, ale nejdostupnější řešení tj. to které už znám, či u hardwaru to 
co mám doma.
