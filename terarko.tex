\documentclass[a4paper,twoside,12pt]{book}
%% === nezbytné balíčky:
\usepackage[T1]{fontenc}    % kódování písma
%\usepackage[IL2]{fontenc}  % kódování písma

\usepackage[utf8]{inputenc}     % vstupní znaková sada tohoto dokumentu: UTF-8
%\usepackage[cp1250]{inputenc}  % vstupní znaková sada tohoto dokumentu: Windows 1250
%\usepackage[latin2]{inputenc}  % vstupní znaková sada tohoto dokumentu: ISO Latin 2

\usepackage[czech]{babel} % česky psaná práce, typografická pravidla. Překládejte pomocí "latex.exe" nebo "pdflatex.exe"
%\usepackage{czech} % česky psaná práce. Překládejte pomocí "pdfCSlatex.exe" ("cslatex.exe" asi bude mít problém 
%s balíkem geometry)
% url
\usepackage[toc]{appendix}
\usepackage[nottoc]{tocbibind}
\usepackage[hyphens]{url}
\renewcommand\appendixtocname{Přílohy}
\renewcommand\appendixpagename{Přílohy}
%pagestyle obsahu
\usepackage{tocloft}
\tocloftpagestyle{obsah}

\usepackage[a4paper, hmarginratio=3:2]{geometry} % využití A4 stránky a nastavení okrajů (u vazby bude širší)

%\usepackage{pdfpages} % pokud nemáte formulář "Zadání bak./dipl. práce" naskenovaný jako PDF, tak ZAKOMENTUJTE
\usepackage[hidelinks]{hyperref} % v PDF budou klikací odkazy ("hidelinks" je nebude rámovat)

%% === balíčky, které se mohou hodit:
%\usepackage{encxvlna} % postará se o spojky a předložky, které dle českých pravidel nesmí být na konci řádku. 
%Dokumentace: http://texdoc.net/texmf-dist/doc/generic/encxvlna/encxvlna.pdf (chová se správně k "vnitřku" listings?)
\usepackage{textcomp}

\usepackage{graphicx} % balíček pro vkládání rastrových grafických souborů (PNG apod.)
\graphicspath{{./images/}}
%\usepackage{epsfig} % balíčky pro vkládání grafických souborů typu EPS
\usepackage{float} % rozšířené možnosti umístění obrázků

\usepackage{caption} % pro popisky obrázků, tabulek atd.

\usepackage{tabularx} % rozšířené možnosti tabulek
%\usepackage{tabu} % jiný balík pro rozšířené možnosti tabulek

\usepackage{listings}  % balíček vhodný pro ukázky zdrojového kódu v~textu práce/příloh. Nutno nastavit! 
%http://ftp.cvut.cz/tex-archive/macros/latex/contrib/listings/listings.pdf
\usepackage{listingsutf8}
\lstset{inputencoding=utf8/latin2}
%\usepackage{amsmath} % balíček pro pokročilou matematickou sazbu
%\usepackage{color} % pro možnost barevného textu
%\usepackage{fancybox} % umožňuje pokročilé rámečkování
\usepackage{fancyhdr} %zahlavi a zapati

%\usepackage{index} % nutno použít v případě tvorby rejstříku balíčkem makeindex
%\newindex{default}{idx}{ind}{Rejstřík} % zavádí rejstřík v případě použití balíku index

%slovníček
\usepackage[toc, nopostdot, acronym, automake]{glossaries}
\makeglossaries
\newglossaryentry{go}
{
  name={Go},
  description={Kompilovaný, staticky typovaný jazyk od Googlu, a na jehož vývoji se podílel například Ken Thompson, 
spolutvůrce programovacího jazyka C, z jehož syntaxe vychází i syntaxe Go. Cíle jazyka jsou zejména jednoduchá syntax, 
strmá křivka učení, či snadná tvorba vícevláknových aplikací. Kompromisem v návrhu jazyka bylo zahrnutí \gls{GC}, 
programy jsou sice pomalejší, ale kód jednodušší. Jazyk je oblíben i mimo Google, je v něm napsán například Docker, či 
ho používá Dropbox.}
}
\newglossaryentry{GC}
{
  name={garbage collector},
  description={Způsob automatické správy paměti. Funguje tak, že speciální algoritmus (garbage collector) vyhledává 
a uvolňuje úseky paměti, které již program nebo proces nepoužívá. Programátor to tedy již nemusí řešit a tím odpadá celá 
řada chyb způsobených zapomenutím na uvolnění paměti\ldots Nevýhodou je, že si garbage collector ukousne část výkonu 
procesoru pro sebe, takže pak program běží pomaleji.}
}
\newglossaryentry{cross-kompilace}
{
  name={cross-kompilace},
  description={Překlad programu pro jinou platformu, než na které je překládán.}
}
\newglossaryentry{linkovani}
{
  name={linkování},
  description={Proces spojení objektového souboru vygenerovaného překladačem s knihovnami, či jinými soubory. Existují 
dva typy dynamické, kdy se knihovny připojují až za běhu a jsou společné pro všechny programy běžící na daném počítači 
a statické, kdy jsou všechny knihovny přibaleny k výslednému spustitelnému souboru.}
}


%biblatex
\usepackage[backend=biber,style=iso-authoryear]{biblatex}
%\bibliographystyle{czplain}
\addbibresource{literatura.bib}

\frenchspacing % za větou bude mezislovní mezera (v anglických textech je mezera za větou delší)
\widowpenalty=10000 % "síla" zákazu vdov (= jeden řádek ze začátku odstavce na konci stránky)
\clubpenalty=10000 % "síla" zákazu sirotků (= jeden řádek/slovo z konce odstavce samostatně na začátku stránky)
\brokenpenalty=10000 % "síla" zákazu zlomu stránky za řádkem, který má na konci rozdělené slovo

\topmargin=-15mm      % horní okraj trochu menší
\textwidth=150mm      % šířka textu na stránce
\textheight=240mm     % "výška" textu na stránce

%záhlaví
\fancypagestyle{plain}
{
	\setlength{\headheight}{12mm}
	\fancyhf{}
	\fancyhead[RO]{Prokop Parůžek}
	\fancyhead[LO]{Monitoring terária}
	\fancyhead[RE]{Gymnázium Teplice}
	\fancyhead[LE]{\includegraphics[height=1cm]{gympl.png}}
	\fancyfoot[C]{\thepage}
}
%záhlaví obsahu
\fancypagestyle{obsah}
{
	\fancyhf{}
	\fancyhead[R]{Prokop Parůžek}
	\fancyhead[L]{Školní rok 2020/2021}
}

\pagenumbering{arabic} % číslování stránek arabskými číslicemi
\pagestyle{plain}      % stránky číslované dole uprostřed

\usepackage{indentfirst}
\parindent=15pt % odsazení 1. řádku odstavce
\parskip=7pt   % mezera mezi odstavci

\newcommand{\ti}{\textit} % zkrácený příkaz pro kurzívu
\newcommand{\tb}{\textbf} % zkrácený příkaz pro tučné písmo


%% --- zde jsou zavedeny některé "konstanty" - některé musíte změnit! --- %%
\newcommand{\cvut}{Gymnázium Teplice, Čs. dobrovolců 11}
\newcommand{\fjfi}{}
\newcommand{\ksi}{}
\newcommand{\program}{} % změňte, pokud máte jiný stud. program
\newcommand{\obor}{Programování a výpočetní technika} % změňte, pokud máte jiný obor

\newcommand{\druh}{Seminární práce} % nebo "Diplomová práce"
\newcommand{\woman}{} % pokud jste ŽENA, ZMĚŇTE na: ...{\woman}{a} (je to do Prohlášení)

\newcommand{\logoCVUT}{\includegraphics[height=8cm]{gympl}}  %logo ČVUT -- podle grafického manuálu 
%ČVUT platného od prosince 2016. Pokud nevyhovuje PDF-verze, tak použijte jinou variantu loga: 
%https://www.cvut.cz/logo-a-graficky-manual -> "Symbol a logo ČVUT v Praze"). Pokud chcete logo úplně vynechat, zadejte 
%místo "\includegraphics{...}" text "\vspace{35mm}"
%\newcommand{\logoCVUT}{\vspace{35mm}}

% přesně podle formuláře "Zadání bak./dipl. práce" VYPLŇTE:
\newcommand{\nazevcz}{Monitoring terária}	% český název práce (přesně podle zadání!)
\newcommand{\nazeven}{Terrarium monitoring}	% anglický název práce (přesně podle zadání!)
\newcommand{\autor}{Prokop Parůžek}   % vyplňte své jméno a příjmení (s akademickým titulem, máte-li jej)
\newcommand{\trida}{7.A} % třída
\newcommand{\vedouci}{Ing. Věra Minaříková} % vyplňte jméno a příjmení vedoucího práce, včetně titulů, např.: Doc. Ing. 
%Ivo Malý, Ph.D.
\newcommand{\pracovisteVed}{} % ZMĚŇTE, pokud vedoucí Vaší práce není z KSI
\newcommand{\konzultant}{} % POKUD MÁTE určeného konzultanta, NAPIŠTE jeho jméno a příjmení
\newcommand{\pracovisteKonz}{} % POKUD MÁTE konzultanta, NAPIŠTE jeho pracoviště
% podle skutečnosti VYPLŇTE:
\newcommand{\rok}{2021}  % rok odevzdání práce (jen rok odevzdání, nikoli celý akademický rok!)
\newcommand{\kde}{Teplicích} % studenti z Děčína ZMĚNÍ na: "Děčíně" (doplní se k "prohlášení")

\newcommand{\klicova}{měření, IoT, }   % zde NAPIŠTE česky max. 5 klíčových slov
\newcommand{\keyword}{measure, IoT, }       % zde NAPIŠTE anglicky max. 5 klíčových slov (přeložte z češtiny)
\newcommand{\abstrCZ}{ Cílem práce je vytvořit automatický systém na sledování teploty a vlhkosti a dalších údajů v teráriu s masožravými rostlinami. Zpřístupnit naměřené údaje online, v podobě grafů, aby uživatel mohl v reálném čase sledovat jak se jeho 
kytičkám daří. Zároveň je kladen důraz na snadnou rozšiřitelnost o další naměřené hodnoty, či o úplně nové senzory, 
místnosti.}    % zde NAPIŠTE abstrakt v češtině (cca 7 vět, min. 80 slov)
\newcommand{\abstrEN}{} % zde NAPIŠTE abstrakt v angličtině

\newcommand{\prohlaseni}{Prohlašuji, že jsem svou seminární práci vypracoval\woman{} samostatně a použil\woman{} jsem pouze podklady (literaturu, projekty, SW atd.) uvedené v seznamu vloženém v práci. 

Prohlašuji, že tištěná verze a elektronická verze práce jsou shodné. 

Nemám závažný důvod proti zpřístupňování této práce v souladu se zákonem č. 121/2000 Sb., o právu autorském, o právech souvisejících s právem autorským a o změně některých zákonů (autorský zákon) v platném znění. } % text prohlášení můžete mírně upravit 
%:-)

\newcommand{\podekovani}{Děkuji ... za ... (upravte makro {\texttt{\textbackslash podekovani\{\}}}).} % NAPIŠTE 
%poděkování, např. svému vedoucímu:
% Děkuji Ing. Eleonoře Krtečkové, Ph.D. za vedení mé bakalářské práce a za podnětné návrhy, které ji obohatily.
% NEBO:
% Děkuji vedoucímu práce doc. Pafnutijovi Snědldítětikaši, Ph.D. za neocenitelné rady a pomoc při tvorbě bakalářské 
% práce.


\begin{document}
%%%%%%%%%%%% TITULNÍ STRANA -- na následujících cca 30 řádků NESAHEJTE!!!  Generuje se AUTOMATICKY %%%%%%%%%%%%
\thispagestyle{empty}

\begin{center}
   {\Large \MakeUppercase{\druh}}
	\vspace{5mm}

	\begin{tabular}{c}
		\tb{\ksi} \\[3pt]   \tb{Předmět: \obor}\\
	\end{tabular}

	\vspace{10mm}
   {\huge \tb{\nazevcz}\par}
   \vspace{5mm}   {\huge \tb{\nazeven}\par}
   
   \vspace{10mm}
	 \begin{tabular}{c}
			 \LARGE{\tb{Škola:} \tb{\cvut}}
	\end{tabular}

   \vspace{10mm} \logoCVUT \vspace{15mm} 

   \vfill
   {\large
	\begin{tabular}{ll}
	Kraj: & Ústecký\\
	Vypracoval: & \autor, \trida\\
	Konzultant: & \vedouci\\
	\end{tabular}
	 }\\
\vspace{1cm}
\LARGE{Teplice 2021}
\end{center}

\clearpage{\pagestyle{empty}\cleardoublepage} % prázdná stránka za tou "titulní", bez čísla

%%%%%%%%%%%% Prohlášení -- SEM NESAHEJTE! Generuje se automaticky z výše nastavených maker \kde{} a \prohlaseni{}. %%%%%%%%%%%%
\newpage % SEM NESAHEJTE!
\thispagestyle{empty}  % SEM NESAHEJTE!

~ % SEM NESAHEJTE!
\vfill % prázdné místo. SEM NESAHEJTE!

\tb{Prohlášení} % SEM NESAHEJTE!

\vspace{1em} % vertikální mezera. SEM NESAHEJTE!
\prohlaseni

\vspace{2em}  % SEM NESAHEJTE!
\hspace{-0.5em}\begin{tabularx}{\textwidth}{X c}  % SEM NESAHEJTE!
V \kde\ dne .................... &........................................ \\	% SEM NESAHEJTE!
	& \autor
\end{tabularx}	% SEM NESAHEJTE!


%%%%%%%%%%%% Poděkování  %%%%%%%%%%%%
\newpage
\thispagestyle{empty}

~
\vfill % prázdné místo


% -- následující kus kódu (do "%%%%%%%%%%%% ABSTRAKT") můžete odstranit, pokud nechcete psát poděkování:
\tb{Poděkování}

\vspace{1em} % vertikální mezera
\podekovani
\begin{flushright}
\autor
\end{flushright}  % <------- tady končí stránka s poděkováním


%%%%%%%%%%%% ABSTRAKT atp. Je generován AUTOMATICKY podle maker nastavených na začátku souboru) %%%%%%%%%%%% 
\newpage   % SEM NESAHEJTE!
\thispagestyle{empty}   % SEM NESAHEJTE!

% příprava:    (na následujících 8 řádků NESAHEJTE!)
\newbox\odstavecbox
\newlength\vyskaodstavce
\newcommand\odstavec[2]{%
	\setbox\odstavecbox=\hbox{%
		 \parbox[t]{#1}{#2\vrule width 0pt depth 4pt}}%
	\global\vyskaodstavce=\dp\odstavecbox
	\box\odstavecbox}
\newcommand{\delka}{120mm} % šířka textů ve 2. sloupci tabulky

% použití přípravy:    % dovnitř "tabular" vůbec NESAHEJTE!
\begin{tabular}{ll}
%  {\em Název práce:} & ~ \\
%	\multicolumn{2}{l}{\odstavec{\textwidth}{\bf \nazevcz}} \\[1em]
	%{\em Autor:} & \autor \\%[1em]
%  {\em Studijní program:} & \program \\
	%{\em Předmět:} & \obor \\
	%{\em Druh práce:} & \druh \\%[1em]
	%{\em Vedoucí práce:} & \odstavec{\delka}{\vedouci\\ \pracovisteVed} \\
%	{\em Konzultant:} & -- %\odstavec{\delka}{\konzultant \\ \pracovisteKonz}  % VYMAŽTE text "-- %" v případě, že jste 
	%	        \\[1em]
	\multicolumn{2}{l}{\odstavec{\textwidth}{{\em Anotace:} ~ \abstrCZ  }} \\[1em]
	\\
  {\em Klíčová slova:} & \odstavec{\delka}{\klicova} \\[2em]

%  {\em Title:} & ~\\
%	\multicolumn{2}{l}{\odstavec{\textwidth}{\bf \nazeven}}\\[1em]
	%{\em Author:} & \autor \\[1em]
	\multicolumn{2}{l}{\odstavec{\textwidth}{{\em Annotation:} ~ \abstrEN  }} \\[1em]
	\\
  {\em Key words:} & \odstavec{\delka}{\keyword}
\end{tabular}



%%%%%%%%%%%% Obsah práce ... je generován AUTOMATICKY %%%%%%%%%%%%
\newpage  % SEM NESAHEJTE!
\parskip=0pt
\tableofcontents % SEM NESAHEJTE!
\parskip=7pt
\newpage % SEM NESAHEJTE!


%--------------------------------------------------------
%|         Zde začíná SAMOTNÁ PRÁCE (text)              |
%--------------------------------------------------------

\chapter*{Úvod} % SEM NESAHEJTE!
\addcontentsline{toc}{chapter}{Úvod} % SEM NESAHEJTE!
%
%Zde napište text úvodu (1-3 strany, nerozdělujte na podkapitoly) nebo jej vložte ze samostatného souboru: např. 
%příkazem \texttt{\textbackslash input\{vnitrek\_uvod.tex\}}.
%
Už potřetí zahajuji svůj pokus pěstovat masožravky, který zatím vždy skončil jejich úhynem. Z toho důvodu jsem se 
rozhodl začít sledovat prostředí v teráriu, kde je pěstuji, abych mohl v případě úhynu určit z jakého důvodu uhynuly. 
Přehřáli se, umrzli, uschly\ldots Většinou z důvodu mé nepřítomnosti, kdy jsem je nemohl kontrolovat. Avšak mnohem radši 
bych byl, kdyby se mi pomocí naměřených údajů podařilo udržet prostředí ve kterém prosperují a v případě náhlé změny 
mohl zasáhnout v krajním případě i vzdáleně.

Cílem mé práce je navržení systému pro měření v podstatě libovolných hodnot, jejich agregování na jednom místě, 
s možností zobrazení aktuálních dat, či jejich průběhu v minulosti, či navázáním různých alarmů na kritické hodnoty. 
Hodnoty by uživatel kontroloval s využítím webové aplikace, které zárověň zajistí snadnou použitelnost na mnoha 
platformách a přístupnost takřka na celém světě, tedy tam kde je internet.

Výsledkem práce bude samotná realizace řešení, od výběru hardwaru a dalších věcí jako je databéze\ldots po samotné 
sestavení měřícího zařízení, jeho naprogramování a naprogramování aplikace na zobrazení naměřených dat. Výsledný produk 
by měl být snadno použitelný a rozšiřitelný o další funkce, možný budoucí vývoj je až aplikace na řízení tzv. chytrého 
domu. Z tohoto důvodu bude kladen důraz i na zabezpečení, pro zamezení neoprávněného přístupu. Z důvodů urychlení 
a zlevnění vývoje, nebudu vždy používat nejvhodnější, ale nejdostupnější řešení tj. to, které už znám, či u hardwaru to 
co mám doma.



%\chapter{Název první kapitoly}
%
%Tady začněte psát text první kapitoly práce nebo jej vložte ze samostatného souboru, např. příkazem 
%\texttt{\textbackslash input\{vnitrek\_kapitola1.tex\}}.
%
\chapter{Požadavky na řešení}
% úvod
%Na celé řešení mám několik požadavků, které postupně detailně popíši tak, aby bylo vše jasné. Poněvadž není nic horšího 
%než nepřesné, či nejasné zadání, protože se pak výsledek špatně hodnotí a celkově upřesňování zadání v průběhu řešení 
%je cesta do pekel. Tyto požadavky jsou nutné, pokud vytvořené řešení bude mít funkce navíc, nevadí to, ale nesmí 
%ovlivňovat tyto zadané parametry.

% měření
První požadavek se bude týkat měření. Měřit chci teplotu a vlhkost v teráriu, když bude zvolený hardware umět i něco 
jiného, klidně to použiji, ale hlavní požadavek je na tyto dvě veličiny. Ohledně frekvence měření chci zachytit denní 
trendy, ale nepotřebuji data z každé minuty.

% ukládání
Další z požadavků je na ukládání dat. Když už je změřím, tak je chci mít vždy uložená, tedy i při výpadku internetu 
a podobně. Při výpadku proudu nic nezměřím, takže to není třeba řešit. Co se týče vzdáleného ukládání, nepovažuji za 
důležité, aby se všechna data propsala do cloudu, tedy i v případě nějakého výpadku se odeslala data co jsem změřil, ale 
neodeslal. Když přijdu o jedno měření během krátkého výpadku, je mi to jedno.% a při větším nebo nějaké chybě si toho 
%všimnu a stejně to bude třeba opravit.

% analýza dat
Základní požadavek je pouze zobrazení dat. Avšak hezká by byla možnost zobrazit si nějaké pokročilejší statistiky. Takže 
volitelně přidávám požadavek na zobrazení různých časových rámců a statistiky k onomu časovému období, například 
průměrná, nejvyšší, nejnižší či medián teploty a vlhkosti. Případně různé tendence, derivace, co bude v možnostech 
vybraného nástroje.

% UI
Nároky na uživatelské rozhraní v podstatě nemám. Pro správu senzorů je nepotřebné. Stejně moc obměňovat, či přidávat 
nebudu, a i kdyby Stejně si budu muset napsat obslužný program pro ten či onen. Udělat nějaký obalující systém pro více 
senzorů, či knihovnu není mým cílem. Pro zobrazení hodnot je důležité, ale ohledně nároků mi bude stačit velmi 
jednoduché, pokud možno jediná stránka. Avšak případnému rozšíření se nebráním. Další části u nichž by bylo třeba 
komunikovat s uživatelem mě nenapadají. Možná správa uživatelů s přístupem, ale vzhledem k tomu, že to dělám pro sebe 
bych to vynechal.

% bezpečnost
K bezpečnosti bych rád zmínil toto. Bylo by dobré mít komunikaci po celé komunikační trase šifrovanou, ale dokud řešení 
neobsahuje ovládání čehosi, není to úplně nezbytné. Případný útočník by si tak maximálně zjistil vlhkost\ldots v teráriu
nebo by se mohl teoreticky vydávat za teploměr a kazit mi data. To by mohl být problém, takže pokud nevyžaduji 
šifrování, ověření toho kdo posílá data požaduji určitě. Zabezpečení však považuji za důležité u samotné webové stránky, 
která bude vystavena veřejně. Ne že by mi vadilo, že někdo sleduje jak se mají mé kytičky, ale mohlo by se z toho dát 
odvodit, že je nezalévám tj. jsem pryč a v bytě nikdo není.% Pokud si někdo dá tu práci, že mi napíchne připojení, nebo 
%přijede a odposlechne data z domácí sítě, tak má asi i možnosti jak si to, že nejsem doma zjistit jinak. Rozhodně to 
%ale bude složitější než otevření webové stránky.

% cena
%Pak tu mám poslední z požadavků a tím je cena. Nejraději bych ho neřešil, ale žijeme ve světě kde je tento požadavek 
%velmi důležitý. Takže asi jediné cenové omezení je, abych si to mohl dovolit. Tady bych zopakoval, že cenu budu 
%částečně upřednostňovat před ostatními požadavky. Zejména v případech, kdy mám nějaký hardware doma, i když ne úplně 
%vhodný, ale dostačující. Tak ho upřednostním před vhodnějším, který nemám a musel bych ho draze shánět. Otázku ceny 
%budu podrobněji řešit u výběru hardwaru.

\chapter{Analýza problému}
% úvod
Zde bych rád rozebral jak budu jednotlivé požadavky řešit. Nastínil jak bude celkové řešení vypadat, strukturu\ldots Též 
bych zde rád představil kde vlastně budu měřit a zpracovávat data.

% měření
Ohledně měření bych zatím zmínil jen frekvence o které myslím myslím, že pro mé účely bude stačit měřit jednou za čtvrt 
hodiny, to je 96 měření za den. Neodchytím tím sice drobné výkyvy v rámci minut, ale cílem je zachytit dlouhodobý průběh 
hodnot v rámci dne, kdy mne tak drobné výkyvy nezajímají. Pro tento účel by se čtvrt hodiny mohlo zdát možná až jako 
příliš často, ale já bych to tak nechal z důvodu zjemnění denních grafů, přeci jenom graf se třeba 12 hodnotami nevypadá 
úplně nejlépe, a také z důvodu odchycení případných chyb měření nebo náhlých změn, například při zalévání atd.

% ukládání
Co se týče uložení dat, abych zajistil stoprocentní jistotu, že se data uloží, budu je ukládat přímo v místě měření, tím 
myslím, že počítač, který bude mít na starosti měřit, bude zároveň naměřená data hned ukládat na SD kartu, nebo tak 
někam. Bude to takový log měření, který mi umožní obnovit data nezapsaná do cloudu z důvodu nějaké chyby, případně mi 
umožní zjišťovat kde vlastně chyba nastala, a může pomoci i s řešením. Dále myslím, že data bude stačit ukládat klidně 
pouze na jedno místo v cloudu, poněvadž ten sám o sobě pokud je kvalitní je výborně zálohovaný, pokud o data v něm 
nějakým nedopatřením přijdu, nějak mě to neovlivní, bude to škoda, avšak závažné důsledky to mít nebude a navíc tím, že 
nebudu řešit zálohy a rozkopírovávání dat, či jejich konzistenci na různých místech se celé řešení výrazně zjednoduší.

% data flow
Teď se dostávám k samotnému data flow, tedy jak a kam mi potečou data co naměřím. Začnu u senzoru, ten změří data 
a pošle je do obslužného počítače, to může být v podstatě cokoli se schopností ovládat senzor,ukládat data a schopností 
poslat data přes wifi. Ten data zaloguje na perzistentní úložiště a pošle je na centrální bránu pomocí wifi sítě, což mi 
přijde nejednoduší, nemusím nikde tahat kabely, či řešit jiné bezdrátové technologie, poněvadž wifi síť má v místě 
měření dostatečné pokrytí. Teď se dostávám k takovému kontroverznímu prvku celého flow, a tím je centrální brána. To je 
v podstatě počítač, který jediný co dělá je přeposílání dat ze senzorů někam jinam, dal by se tedy úplně odstranit s tím 
že měřící stanice by data odesílala přímo do internetu, ale já ji zahrnul z těchto důvodů. Díky tomu, že s internetem 
komunikuje pouze jedna stanice nemusím na ostatních řešit jejich autentizaci vůči cloudu, či různé SSL certifikáty 
a podobně. To mi umožní na jejich místech mohu nasadit mnohem jednoduší zařízení. Dále mi to umožní přístup k datům doma 
i bez internetu, kdybych si je chtěl nějak zobrazit... A Také to zjednoduší ladění celého systému, kdy například 
programy mohu testovat u sebe na počítači kdy data budu brát z centrální brány a nebudu muset vůbec zasahovat do kódu 
v senzoru. No a to je celé z brány pošlu data do cloudu a tam jejich cesta končí, pokud tedy vynechám cestu z cloudu za 
účelem jejich zobrazení.
\begin{figure}[h]
		\centering
		\includegraphics[width=\textwidth]{dataFlow.png}
		\caption{Takto přibližně potečou data}
\end{figure}

% UI + analýza
Uživatelské rozhraní pro zobrazení hodnot jsem se rozhodl z důvodu co největší přenositelnosti ji implementovat jako 
webovou stránku. Takže celá jeho logika a vykreslování bude řešená v javascriptu a až na klientském zařízení, cloud 
použiji pouze na to, abych z něj vytáhl potřebná data a samozřejmě na hostování celé aplikace. Na této stránce určitě 
zobrazím graf vývoje změřených hodnot, myslím že v základu by mohlo stačit tak posledních 48 hodin, ale asi bych přidal 
možnost i delších časových úseků. Z dalších údajů bych si zobrazil tak možná medián naměřených hodnot a možná průměr, 
ale další údaje mi už přijdou zbytečné. Další analýzy dělat nebudu, spíše  bude sledovat jak se kytičkám daří a případně 
je po nasbírání zkušeností doplním.

% bezpečnost
Co se otázky bezpečnosti týče, tak vynechám šifrování na cestě od senzoru do brány, zejména z důvodu jednoduššího ladění 
a implementace, avšak co na této cestě doplním je elektronický podpis zprávy, aby se mi nikdo nefušoval do komunikace. 
Zabezpečení komunikace s cloudem už budu řešit prostředky té dané služby i co se zobrazení hodnot týče.

\chapter{Hardware}
\chapter{Software}
\section{Měřící stanice}
\section{Domácí gateway}
\section{Cloud}
\section{Zobrazení grafů}
\chapter{Výsledek}
 % text vkládán ze souboru. Pokud je v souboru uveden i příkaz \chapter{...}, tak ho 
%o 4 řádky výše vymažte.


\chapter*{Závěr} % SEM NESAHEJTE!
\addcontentsline{toc}{chapter}{Závěr} % SEM NESAHEJTE!
%
%Zde napište text úvodu (1-3 strany, nerozdělujte na podkapitoly) nebo jej vložte ze samostatného souboru: např. 
%příkazem \texttt{\textbackslash input\{vnitrek\_zaver.tex\}}.
%
Hlavní cíl mé práce tj. udržet kytičky naživu se myslím, podařilo splnit, neb zatím žijí. Uvidím jak se jim bude dařit 
dál. Myslím že měření mi v lecčem pomohlo udržet je naživu, například jsem díky němu zjistil, že je málo zalévám, zvýšil 
jsem frekvenci a začali vypadat lépe.

Ohledně splnění cílů myslím, že vytvořené řešení splňuje až na jeden všechny hlavní požadavky, které jsme si zadal. 
Hodnoty se měří, ukládají se je na SD kartu a přes \glslink{brana}{bránu} se posílají do \gls{firebase}, odkud si je 
mohu zobrazit pouze já. Jediný požadavek, jež jsem nezvládl splnit, je ten na podepisování posílaných dat, na ochranu 
před podvržením.

Myslím, že případné zlepšování by se mohlo zaměřit na vylepšení zabezpečení cesty od senzorů na \glslink{brana}{bránu}, 
přidání statistik a možnosti zobrazení hodnot z většího intervalu či zobrazení stavu senzorů přímo u terária. Případně 
by se dalo zrychlit přidávání nových senzorů, pokud možno tak, že ho přidám, jednou nastavím, kde je a co měří a to bude 
vše.



%%%%%%%%%%%% SEZNAM POUŽITÝCH ZDROJŮ (LITERATURA) %%%%%%%%%%%%
\clearpage  % SEM NESAHEJTE!
%\addcontentsline{toc}{chapter}{Literatura} % SEM NESAHEJTE!

%\begin{thebibliography}{99}   formát: ČSN ISO 690. Můžete si to vygenerovat na http://www.citacepro.com (přihlaste se 
%  přes odkaz "ČVUT"), umí to vygenerovat TeX
% řazení: abecedně podle autora (resp. prvního slova, není-li znám autor)
%\bibitem{odkaz} Autor. \ti{Název knihy}. Město. Nakladatelství. Rok.
%\end{thebibliography}
\nocite{*}
\printbibliography[
	heading=bibintoc,
	title={Literatura}
]
\newpage

\tocloftpagestyle{plain}
%seznam obrázků
\listoffigures
%seznam tabulek
\listoftables
\newpage

% Slovníček pojmů
\printglossary[title={Slovníček pojmů}]

% Akronymy
\printglossary[type=\acronymtype, title=Akronymy]

%%%%%%%%%%%% PŘÍLOHY PRÁCE %%%%%%%%%%%%
\newpage % SEM NESAHEJTE!
%\addcontentsline{toc}{chapter}{Přílohy} % SEM NESAHEJTE!
\appendix % SEM NESAHEJTE!
%\addappheadtotoc
%\appendixpage


%%%%%%%%%%%% Příloha A (tj. 1. kapitola v rámci příloh) %%%%%%%%%%%%
%Zde napište text první přílohy nebo jej vložte, např.: \texttt{\textbackslash input\{priloha\_A.tex\}}.
%
\chapter{Zdrojový kód}
%\lstinputlisting[language=Python]{dallas.py}
 % text vkládán ze souboru, kde je i příkaz \chapter{...}


\end{document} % SEM NESAHEJTE! Konec.
